% Options for packages loaded elsewhere
\PassOptionsToPackage{unicode}{hyperref}
\PassOptionsToPackage{hyphens}{url}
%
\documentclass[
]{article}
\usepackage{amsmath,amssymb}
\usepackage{lmodern}
\usepackage{iftex}
\ifPDFTeX
  \usepackage[T1]{fontenc}
  \usepackage[utf8]{inputenc}
  \usepackage{textcomp} % provide euro and other symbols
\else % if luatex or xetex
  \usepackage{unicode-math}
  \defaultfontfeatures{Scale=MatchLowercase}
  \defaultfontfeatures[\rmfamily]{Ligatures=TeX,Scale=1}
\fi
% Use upquote if available, for straight quotes in verbatim environments
\IfFileExists{upquote.sty}{\usepackage{upquote}}{}
\IfFileExists{microtype.sty}{% use microtype if available
  \usepackage[]{microtype}
  \UseMicrotypeSet[protrusion]{basicmath} % disable protrusion for tt fonts
}{}
\makeatletter
\@ifundefined{KOMAClassName}{% if non-KOMA class
  \IfFileExists{parskip.sty}{%
    \usepackage{parskip}
  }{% else
    \setlength{\parindent}{0pt}
    \setlength{\parskip}{6pt plus 2pt minus 1pt}}
}{% if KOMA class
  \KOMAoptions{parskip=half}}
\makeatother
\usepackage{xcolor}
\usepackage[margin=1in]{geometry}
\usepackage{graphicx}
\makeatletter
\def\maxwidth{\ifdim\Gin@nat@width>\linewidth\linewidth\else\Gin@nat@width\fi}
\def\maxheight{\ifdim\Gin@nat@height>\textheight\textheight\else\Gin@nat@height\fi}
\makeatother
% Scale images if necessary, so that they will not overflow the page
% margins by default, and it is still possible to overwrite the defaults
% using explicit options in \includegraphics[width, height, ...]{}
\setkeys{Gin}{width=\maxwidth,height=\maxheight,keepaspectratio}
% Set default figure placement to htbp
\makeatletter
\def\fps@figure{htbp}
\makeatother
\setlength{\emergencystretch}{3em} % prevent overfull lines
\providecommand{\tightlist}{%
  \setlength{\itemsep}{0pt}\setlength{\parskip}{0pt}}
\setcounter{secnumdepth}{-\maxdimen} % remove section numbering
\ifLuaTeX
  \usepackage{selnolig}  % disable illegal ligatures
\fi
\IfFileExists{bookmark.sty}{\usepackage{bookmark}}{\usepackage{hyperref}}
\IfFileExists{xurl.sty}{\usepackage{xurl}}{} % add URL line breaks if available
\urlstyle{same} % disable monospaced font for URLs
\hypersetup{
  pdftitle={Thesis Writeup},
  pdfauthor={Julian Sim},
  hidelinks,
  pdfcreator={LaTeX via pandoc}}

\title{Thesis Writeup}
\author{Julian Sim}
\date{}

\begin{document}
\maketitle

\hypertarget{chapter-1-introduction}{%
\section{Chapter 1: Introduction}\label{chapter-1-introduction}}

The increasing rate of epidemics has posed a challenge for various
schools, especially for districts which do not have the necessary
resources to transition to online education. Although the educational
benefits of different bell schedules have been studied, the purpose of
this study was to determine whether alternative schedules have the
potential to reduce the rate at which epidemics spread throughout
schools. In order to conduct this study, we expanded traditional
compartmental models to utilize network structures with an extension of
the EpiModel R package. Utilizing an anonymized network dataset of
secondary school friendships alongside the previous tools, we were able
to determine that there is no measurable difference between the standard
8 period schedule and A/B block schedules.

\hypertarget{chapter-2-literature-review}{%
\section{Chapter 2: Literature
Review}\label{chapter-2-literature-review}}

\hypertarget{compartmental-models}{%
\subsection{Compartmental Models}\label{compartmental-models}}

\hypertarget{networks}{%
\subsection{Networks}\label{networks}}

\hypertarget{chapter-3-datasets-faux.mesa.high}{%
\section{Chapter 3: Datasets
(faux.mesa.high)}\label{chapter-3-datasets-faux.mesa.high}}

The primary dataset utilized within this study is Goodreau's Faux Mesa
High School, often referred to as faux.mesa.high. This dataset was
created by fitting data from the first wave of the Add Health Study, a
national survey of adolescent health. After the student information was
anonymized via random permutation of student attributes, an ERGM model
was fit to the data, which was then applied to a network with vertices
corresponding to each anonymized student.

\hypertarget{chapter-4-methodology-with-example}{%
\section{Chapter 4: Methodology (with
example)}\label{chapter-4-methodology-with-example}}

\hypertarget{epimodel}{%
\subsection{EpiModel}\label{epimodel}}

\hypertarget{schedule-generation}{%
\subsection{Schedule Generation}\label{schedule-generation}}

\hypertarget{ergm}{%
\subsection{ERGM}\label{ergm}}

\hypertarget{chapter-5-modeling}{%
\section{Chapter 5: Modeling}\label{chapter-5-modeling}}

\hypertarget{chapter-6-shiny-with-example}{%
\section{Chapter 6: Shiny (with
example)}\label{chapter-6-shiny-with-example}}

\hypertarget{chapter-7-conclusions-future-work}{%
\section{Chapter 7: Conclusions \& Future
Work}\label{chapter-7-conclusions-future-work}}

\end{document}
